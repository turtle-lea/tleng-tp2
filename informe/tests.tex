\lstset{
  frame=none,
  xleftmargin=2pt,
  stepnumber=1,
  numbers=left,
  numbersep=5pt,
  numberstyle=\ttfamily\tiny\color[gray]{0.3},
  belowcaptionskip=\bigskipamount,
  captionpos=b,
  escapeinside={*'}{'*},
  language=prolog,
  tabsize=2,
  emphstyle={\bf},
  commentstyle=\it,
  stringstyle=\mdseries\rmfamily,
  showspaces=false,
  keywordstyle=\bfseries\rmfamily,
  columns=flexible,
  basicstyle=\small\sffamily,
  showstringspaces=false,
  morecomment=[l]\%,
}

\subsection{Tests con fallas}

\subsubsection{Test 1: Tiene compaces con distinta duración}

\begin{lstlisting}
#tempo negra 30
#compas 3/4

const oct1 = 2;
const oct2 = 6;
const oct3 = 1;

// Instrumentos
const flauta = 51;

voz (flauta)
{
  compas
  {
    nota(do, oct3, blanca.);
    nota(re, oct1, redonda);
  }

  compas
  {
    nota(mi, oct2, blanca);
    nota(la, oct1, negra);
  }
}
\end{lstlisting}
\vspace{5mm}

\subsubsection{Test 2: Tiene voces con compaces de distinta duración}

\begin{lstlisting}
#tempo negra 120
#compas 2/2

const oct1 = 5;
const oct2 = 2;
const oct3 = 4;

// Instrumentos
const violin = 20;
const guitarra = 12;

voz (violin)
{
  compas
  {
    nota(do, oct3, blanca.);
    nota(re, oct1, negra);
  }

  compas
  {
    nota(mi, oct2, blanca.);
    nota(la, oct1, negra);
  }
}

voz (guitarra)
{
  compas
  {
    nota(do, oct3, fusa);
    nota(re, oct1, semifusa.);
  }

  compas
  {
    nota(mi, oct2, fusa);
    nota(la, oct1, semifusa.);
  }
}
\end{lstlisting}
\vspace{5mm}

\subsubsection{Test 3: Constante que apunta a una constante no definida}

\begin{lstlisting}
#tempo negra 60
#compas 1/1

const oct1 = 3;
const oct2 = ConstanteTrucha;

// Instrumentos
const bajo = 20;

voz (bajo)
{
  compas
  {
    nota(do, oct1, blanca.);
    nota(re, oct1, negra);
  }

  compas
  {
    nota(mi, oct2, blanca.);
    nota(la, oct2, negra);
  }
}
\end{lstlisting}
\vspace{5mm}

\subsubsection{Test 4: Constante definida circularmente}

\begin{lstlisting}
#tempo negra 60
#compas 2/8

const oct1 = 3;
const oct2 = 5;

// Instrumentos
const bajo = 20;
const malPensado = malPensado;

voz (bajo)
{
  compas
  {
    nota(do, oct1, corchea.);
    nota(re, oct1, semicorchea);
  }

  compas
  {
    nota(mi, oct2, semicorchea);
    nota(la, oct2, corchea.);
  }
}
\end{lstlisting}
\vspace{5mm}

\subsection{Tests correctos}

\subsubsection{Test 1: Simple}

\begin{lstlisting}
#tempo negra 30
#compas 2/4

const oct1 = 2;
const oct2 = 6;
const oct3 = 1;

// Instrumentos
const flauta = 51;

voz (flauta)
{
  compas
  {
    nota(do, oct3, negra);
    nota(re, oct1, negra);
  }

  compas
  {
    nota(mi, oct2, blanca);
  }
}
\end{lstlisting}
\vspace{5mm}

\subsubsection{Test 2: Con varias voces}

\begin{lstlisting}
#tempo negra 120
#compas 3/4

const oct1 = 2;
const oct2 = 6;
const oct3 = 1;
const oct4 = 3

// Instrumentos
const piano = 65;
const violin = 31;

voz (piano)
{
  compas
  {
    nota(sol, oct3, blanca);
    nota(fa+, oct2, negra);
  }

  compas
  {
    nota(mi, oct2, negra.);
    nota(fa, oct4, corchea);
    nota(mi, oct2, negra);
  }
  compas
  {
    silencio(negra);
    nota(sol-, oct1, negra);
    nota(sol-, oct1, negra);
  }
}

voz (violin)
{
  compas
  {
    nota(la, oct1, blanca.);
  }
  
  compas
  {
    nota(mi, oct2, negra.);
    nota(fa, oct4, corchea);
    nota(mi, oct2, negra);
  }
  compas
  {
    silencio(semicorchea);
    nota(mi, oct1, semicorchea);
    nota(sol, oct4, corchea);
    nota(sol, oct4, blanca);
  }
}
\end{lstlisting}
\vspace{5mm}

\subsubsection{Test 3: Con repeticiones}

\begin{lstlisting}
#tempo negra 120
#compas 2/4

const oct1 = 2;
const oct2 = 6;
const oct3 = 1;
const oct4 = 3

// Instrumentos
const flauta = 10;
const violin = 31;

voz (flauta)
{
  repetir (5)
  {
  	compas
  	{
    		nota(sol, oct3, negra);
    		nota(fa+, oct2, negra);
  	}

  	compas
  	{
    		nota(mi, oct2, negra.);
    		nota(fa, oct4, corchea);
  	}
  	compas
  	{
    		nota(sol, oct1, blanca);
  	}
  }
}

voz (violin)
{
  compas
  {
    nota(la, oct1, blanca);
  }
  
  compas
  {
    nota(mi, oct2, negra.);
    nota(fa, oct4, corchea);
    nota(mi, oct2, negra);
  }
  compas
  {
    silencio(semifusa);
    nota(si+, oct1, semifusa);
    nota(si, oct4, semifusa);
    nota(fa, oct4, semifusa);
    nota(sol, oct4, semicorchea);
    nota(mi-, oct4, corchea);
    nota(re, oct4, negra);
  }
}
\end{lstlisting}
\vspace{5mm}