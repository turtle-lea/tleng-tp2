El objetivo de nuestro tp es, dado un archivo de entrada, parsearlo para poder tener un archivo de salida, respetando restricciones solicitadas para el correcto funcionamiento.

\subsection{Paso a paso introductivo de la resolución}
\begin{itemize}
\item Nos dan un lenguaje para descripción de partituras \url{www.dc.uba.ar/materias/tl/2015/c1/tp2-enunciado-compositor-musical/at_download/file}.
\item Definimos una gramática para el lenguaje.
\item Utilizamos la herramienta PLY configurada con nuestra gramática para generar un AST (árbol sintáctico)
\item En base al árbol generado y validado, escribiremos un archivo de salida con el formato especificado en \url{www.dc.uba.ar/materias/tl/2015/c1/tp2-enunciado-compositor-musical/at_download/file}.
\end{itemize}
%\begin{itemize}
%\item En primera instancia definiremos una gramática necesaria para interpretar, de manera correcta, nuestro archivo de entrada.
%\item Luego definiremos cada expresion regular como tokens en el lexer\_rules, de esta manera cargamos en la computadora nuestras expresiones regulares definidas en la gramática.
%\item El siguiente paso será definir las producciones correspondiente, las cuales gracias a los tokens definidos en el paso anterior, podremos construir, diferenciarlas una de las otras y filtrar aquellas que no sean válidas.
%\item Cada producción llamará a su función interna, formando el árbol decado desde las hojas hasta su raiz, usando en cada una atributos sintetizados para poder intercambiar valores de una rama a la otra y poder validar las condiciones especificadas en nuestro lenguaje.
%\item Una vez corroborado y habiendo obtenido con éxito todo lo anterior, procederemos a escribir el archivo MIDI cumpliendo las normas del mismo sin dificultades.
%\end{itemize}
