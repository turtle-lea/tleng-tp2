El objetivo de nuestro tp es, dado un archivo de entrada, parsearlo para poder tener un archivo de salida, respetando restricciones solicitadas para el correcto funcionamiento.

\subsection{Paso a paso introductivo de la resolución}
\begin{itemize}
\item En primera instancia definiremos una gramática necesaria para interpretar, de manera correcta, nuestro archivo de entrada. 
\item Luego definiremos cada expresion regular como tokens en el lexer\_rules, de esta manera cargamos en la computadora nuestras expresiones regulares definidas en la gramática.
\item El siguiente paso será definir las producciones correspondiente, las cuales gracias a los tokens definidos en el paso anterior, podremos construir, diferenciarlas una de las otras y filtrar aquellas que no sean válidas.
\item Cada producción llamará a su función interna, formando el árbol decado desde las hojas hasta su raiz, usando en cada una atributos sintetizados para poder intercambiar valores de una rama a la otra y poder validar las condiciones especificadas en nuestro lenguaje.
\item Una vez corroborado y habiendo obtenido con éxito todo lo anterior, procederemos a escribir el archivo MIDI cumpliendo las normas del mismo sin dificultades.
\end{itemize}