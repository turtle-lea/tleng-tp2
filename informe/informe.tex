\documentclass[a4paper, 10pt, twoside]{article}

\usepackage[top=1in, bottom=1in, left=1in, right=1in]{geometry}
\usepackage[utf8]{inputenc}
\usepackage[spanish, es-ucroman, es-noquoting]{babel}
\usepackage{setspace}
\usepackage{fancyhdr}
\usepackage{lastpage}
\usepackage{amsmath}
\usepackage{amsfonts}
\usepackage{amsthm}
\usepackage{verbatim}
\usepackage{graphicx}
\usepackage{float}
\usepackage{enumitem} % Provee macro \setlist
\usepackage{tabularx}
\usepackage{multirow}
\usepackage{hyperref}
\usepackage{multicol}
\usepackage[toc, page]{appendix}
\usepackage{ulem} % para subrayados especiales
\usepackage{color}
\usepackage{framed}
\usepackage{algorithm}
\usepackage{algpseudocode}
\usepackage{graphicx}
\usepackage{float}
\usepackage{listings}
\usepackage{hyperref}

\definecolor{shadecolor}{rgb}{0.95,0.95,0.95}


%%%%%%%%%% Configuración de Fancyhdr - Inicio %%%%%%%%%%
\pagestyle{fancy}
\thispagestyle{fancy}
\lhead{Trabajo Práctico 2 · Teoría de Lenguajes}
\rhead{Matayoshi · Panarello · Vega}
\renewcommand{\footrulewidth}{0.4pt}
\cfoot{\thepage /\pageref{LastPage}}

\fancypagestyle{caratula} {
   \fancyhf{}
   %\cfoot{\thepage /\pageref{LastPage}}
   \renewcommand{\headrulewidth}{0pt}
   \renewcommand{\footrulewidth}{0pt}
}
%%%%%%%%%% Configuración de Fancyhdr - Fin %%%%%%%%%%


%%%%%%%%%% Configuración de Hyperref - Inicio %%%%%%%%%%
\hypersetup{%
 % Para que el PDF se abra a pagina completa.
  pdfstartview= {FitH \hypercalcbp{\paperheight-\topmargin-1in-\headheight}},
  pdfauthor={Matayoshi,Panarello,Vega},
  pdfsubject={Tleng-TP1},
 %pdfkeywords={keyword1} {key2} {key3},
 colorlinks=true,
  linkcolor=black,
  urlcolor=blue
}
%%%%%%%%%% Configuración de Hyperref - Fin %%%%%%%%%%


%%%%%%%%%% Miscelánea - Inicio %%%%%%%%%%
% Evita que el documento se estire verticalmente para ocupar el espacio vacío
% en cada página.
\raggedbottom

% Deshabilita sangría en la primer línea de un párrafo.
\setlength{\parindent}{0em}

% Separación entre párrafos.
\setlength{\parskip}{0.5em}

% Separación entre elementos de listas.
\setlist{itemsep=0.5em}

% Asigna la traducción de la palabra 'Appendices'.
\renewcommand{\appendixtocname}{Apéndices}
\renewcommand{\appendixpagename}{Apéndices}
%%%%%%%%%% Miscelánea - Fin %%%%%%%%%%


%%%%%%%%%% Insertar diagrama - Inicio %%%%%%%%%%
\newcommand{\diagramav}[1]{
  \includegraphics[type=png,ext=.png,read=.png,width=16cm]{#1}
}

\newcommand{\diagramah}[1]{
  \includegraphics[type=png,ext=.png,read=.png,height=12.8cm,angle=90]{#1}
}
%%%%%%%%%% Insertar diagrama - Fin %%%%%%%%%%


\begin{document}
\pagenumbering{gobble}


%%%%%%%%%%%%%%%%%%%%%%%%%%%%%%%%%%%%%%%%%%%%%%%%%%%%%%%%%%%%%%%%%%%%%%%%%%%%%%%
%% Carátula                                                                  %%
%%%%%%%%%%%%%%%%%%%%%%%%%%%%%%%%%%%%%%%%%%%%%%%%%%%%%%%%%%%%%%%%%%%%%%%%%%%%%%%


\thispagestyle{caratula}

\begin{center}

\includegraphics[height=2cm]{caratula/DC.png}
\hfill
\includegraphics[height=2cm]{caratula/UBA.jpg}

\vspace{2cm}

Departamento de Computación,\\
Facultad de Ciencias Exactas y Naturales,\\
Universidad de Buenos Aires

\vspace{4cm}

\begin{Huge}
Trabajo Práctico 2
\end{Huge}

\vspace{0.5cm}

\begin{huge}
Compositor Musical
\end{huge}

\vspace{0.5cm}

\begin{Large}
Teoría de Lenguajes
\end{Large}

\vspace{1cm}

Primer Cuatrimestre de 2015

\vspace{3.5cm}

\begin{Large}
Grupo: Autores del Autómata Automático Autodestructivo
\end{Large}

\vspace{0.5cm}

\begin{tabular}{|c|c|c|}
\hline
Apellido y Nombre & LU & E-mail\\
\hline
Matayoshi, Leandro  & 79/11 & {\tt leandro.matayoshi@gmail.com}\\
Panarello, Bernabé  & 194/01 & {\tt bpanarello@gmail.com}\\
Vega, Leandro    & 698/11 & {\tt leandrogvega@gmail.com}\\
\hline
\end{tabular}

\end{center}

\newpage
\pagenumbering{arabic}

%%%%%%%%%%%%%%%%%%%%%%%%%%%%%%%%%%%%%%%%%%%%%%%%%%%%%%%%%%%%%%%%%%%%%%%%%%%%%%%
%% Índice                                                                    %%
%%%%%%%%%%%%%%%%%%%%%%%%%%%%%%%%%%%%%%%%%%%%%%%%%%%%%%%%%%%%%%%%%%%%%%%%%%%%%%%


\tableofcontents

\newpage


%%%%%%%%%%%%%%%%%%%%%%%%%%%%%%%%%%%%%%%%%%%%%%%%%%%%%%%%%%%%%%%%%%%%%%%%%%%%%%%
%% Descripción del informe			                                        %%
%%%%%%%%%%%%%%%%%%%%%%%%%%%%%%%%%%%%%%%%%%%%%%%%%%%%%%%%%%%%%%%%%%%%%%%%%%%%%%%

\section{Introducción del problema a resolver}
El objetivo de nuestro tp es, dado un archivo de entrada, parsearlo para poder tener un archivo de salida, respetando restricciones solicitadas para el correcto funcionamiento.

\subsection{Paso a paso introductivo de la resolución}
\begin{itemize}
\item En primera instancia definiremos una gramática necesaria para interpretar, de manera correcta, nuestro archivo de entrada.
\item Luego definiremos cada expresion regular como tokens en el lexer\_rules, de esta manera cargamos en la computadora nuestras expresiones regulares definidas en la gramática.
\item El siguiente paso será definir las producciones correspondientes, las cuales gracias a los tokens definidos en el paso anterior, podremos construir, diferenciarlas una de las otras y filtrar aquellas que no sean válidas.
\item Cada producción llamará a su función asociada, formando de esta manera un árbol AST, desde las hojas hasta su raiz, usando en cada una atributos sintetizados para poder intercambiar valores de una rama a la otra y poder validar las condiciones especificadas en nuestro lenguaje y el enunciado.
\item Una vez corroborado y habiendo obtenido con éxito todo lo anterior, procederemos a escribir el archivo MIDI de acuerdo al formato del mismo sin dificultades.
\end{itemize}

\newpage

\section{Descripción del problema resuelto}
En esta sección explicaremos cada parte implementada para realizar los procedimientos requeridos. Contaremos dudas, errores que fueron surgiendo y explicaremos las decisiones tomadas. Para eso vamos a dividirlo en cuatro secciones que detallamos a continuación ordenadas de las formas en las que lo fuimos realizando.

\subsection{Gramática}
No contamos con demasiadas dificultades, miramos cada paso de la descripción de la partitura y fuimos creando las producciones necesarias. Se explicita en la sección Gramática.

\subsection{PLY}
En nuestra implementación utilizamos las herramientas brindadas por PLY como explicamos en la introducción, estas son:
\subsubsection{Lexer}
Vamos a utilizar un lexer para poder tener definidas nuestras expresiones regulares y poder decidir que cadenas son o no válidas según nuestra gramática.
Para generar un lexer vamos a realizar lo descripto en el siguiente link \url{www.dc.uba.ar/materias/tl/2015/c1/files/tp2-clase-intro-a-ply/at_download/file}.
Crearemos un archivo lexer\_rules.py, definiendo los tokens y las reglas.
La lista de tokens será explicitada en la sección de la gramática.\newline
Luego pasamos a definir las reglas para cada expresión regular que deseamos tener. En esta parte tuvimos bastantes problemas, principalmente porque optamos por usar reglas 'simples' en todas las reglas.\newline

El primero de los problemas fue al hacer la siguiente expresión regular "$(do|re|mi|fa|sol|la|si)(+|-)?$". Al tenerla toda junta, la regla matcheaba con todas las notas hasta el final del primer paréntesis sin mirar lo que continuaba y, en caso de tener en el archivo de entrada una nota con "$sol+$", buscaba una regla que inicialice con un $+$ o $-$, la cual no existe, y por lo tanto producía un error. La solución fue separarlas, pudiendo solucionar el problema mencionado. \newline

El segundo problema que surgió fue el tema del orden, al tener reglas como const que generan todo el alfabeto, nos matcheaba voz, compas, entre otras, cuando nosotros en realidad queriamos que esas palabras matcheen en otra regla definida. Como nosotros teníamos definido reglas 'simples', al tratar de ordenar las reglas notamos que la lógica de la clase re (regular expression) tomaba como primera a la que definía en su regla el string más largo, esto nos hizo rever la forma de definir las reglas. Mirando y testeando la clase re, pudimos corroborar que, definiendo las reglas como funciones, se respetaba el orden en las que eran definidas en el script. De esta manera pasamos todas las reglas a funciones como se explica en el pdf del link para reglas 'complejas', logrando salvar dicho problema.\newline

Aún habiendo solucionado los problemas anteriores, surgió un tercer problema. En este caso, se desató un problema con las palabras reservadas. Dado que establecimos un orden previamente, habíamos optado por dejar a la regla $cname$ (que es nuestro constructor de constantes) en el último lugar, y que reglas que usaban palabras como $re$, $silencio$, entre otras, se lograra matchear más arriba. Esto produjo que, al definir constantes con el nombre $fantastico$, y al tener reservada la palabra $fa$, nos imposibilitara usarla produciendo un error. Investigando un poco sobre PLY, logramos encontrar el siguiente enlace \url{http://www.dabeaz.com/ply/ply.html}, el cual habla del problema mencionado y describe como corregirlo. La solución consiste en no definir las palabras reservadas como reglas separadas, sino todo junto dentro de, en nuestro caso, $cname$. De esta manera, toda palabra que inicialice con un caracter del alfabeto va entrar por la regla $cname$, detectará la palabra que tenga que detectar y derivará el token correspondiente segun nuestra lista de $reserved$. Se puede apreciar fácilmente que en nuestro ejemplo mencionado, $fantastico$ sólo va a ser derivado a un token $CONST$ mientras que $fa$ será derivado a un token $NOTENAME$, solucionando el problema descripto.

\subsubsection{Parser}
Vamos a utilizar un parser para poder darle una estructura y una semántica a nuestra grámatica.
Para generar un parser vamos a realizar lo descripto en el link mencionado en el lexer desde la página 15 en adelante. Creamos un archivo parser\_rules.py, en el cual vamos a definir las producciones de la gramática y cómo generar el arbol AST (abstract Syntax Tree). Este archivo usará los tokens de lexer\_rules para construir las producciones, diferenciarlas una de las otras y filtrar aquellas que no sean válidas.\newline
Cada producción llamará a su función interna, que estará definida en parserobject.py, formando el árbol AST desde las hojas hasta su raiz. Cada una de ellas creará un objeto nodo y usará atributos sintetizados para poder intercambiar valores de una rama a la otra y poder validar las condiciones especificadas en nuestro lenguaje, en otras palabras le estaremos dando una semántica a nuestras producciones.
\newpage

\section{Gramática}
%\subsection{Gramática derivada del enunciado}
%
%H $\rightarrow$ Tempo Compasheader (Constinit)? Voicelist\newline
%
%Tempo $\rightarrow$ \# tempo Shape Num\newline
%
%Compasheader $\rightarrow$ \# compas Num / Num\newline
%
%Constinit $\rightarrow$ Constlist\newline
%
%Constlist $\rightarrow$ (Const)$^+$\newline
%
%Const $\rightarrow$ const Value = Value ;\newline
%
%Voicelist $\rightarrow$ (Voice)$^+$\newline
%
%Voice $\rightarrow$ voz ( Value ) \{ VoiceContent \}\newline
%
%Voicecontent $\rightarrow$ (Compas$|$Compasloop)$^+$\newline
%
%Compasloop $\rightarrow$ repetir ( Value ) \{ Compaslist \}\newline
%
%Compas $\rightarrow$ compas \{ Notelist \}\newline
%
%Notelist $\rightarrow$ (Note$|$Silence)$^+$\newline
%
%Note $\rightarrow$ nota ( Notename (+$|$-)? , Value , Shape (.)? ) ;\newline
%
%Value $\rightarrow$ Cname$|$Num\newline
%
%Silence $\rightarrow$ silencio ( Shape (.)? ) ;\newline
%
%Shape $\rightarrow$ blanca$|$negra$|$redonda$|$semicorchea$|$corchea$|$semifusa$|$fusa \newline
%
%Notename $\rightarrow$ do$|$re$|$mi$|$fa$|$sol$|$la$|$si \newline
%
%Cname $\rightarrow$ ([a-z] $|$ [A-Z]) ([0-9] $|$ [a-z] $|$ [A-Z])$^*$\newline
%
%Num $\rightarrow$ [0] $|$ ([1-9] [0-9])$^*$\newline

\subsection{Gramática deducida}

\subsubsection{Tupla}

$G = (V_{t}, V_{n}, P, H)$ 

\subsubsection{Conjunto finito de terminales ($V_{t}$)}

\{ \#tempo, \#compas, $/$, ';', ',', $=$, (, ), \{, \}, '.', $+$, $-$, const, voz, compas, repetir, nota, silencio, blanca, negra, redonda, semicorchea, corchea, fusa, semifusa, do, re, mi, fa, sol, la, si \}

\subsubsection{Conjunto finito de no terminales ($V_{n}$)}

\{ H, TEMPO, COMPASHEADER, CONSTINIT, CONSTLIST, CONST, VOICELIST, VOICE, VOICECONTENT, COMPASLOOP, COMPASLIST, COMPAS, NOTELIST, NOTE, SILENCE, VALUE, SHAPE, NUM, CNAME, NOTENAME, ALTER \}

\subsubsection{Producciones ($P$)}

H $\rightarrow$ \{TEMPO\}\{COMPASHEADER\}\{CONSTINIT\}\{VOICELIST\}\newline
H $\rightarrow$ \{TEMPO\}\{COMPASHEADER\}\{VOICELIST\}\newline
TEMPO $\rightarrow$ \{tempobegin\}\{shape\}\{num\}\newline
COMPASHEADER $\rightarrow$ \{compasheaderbegin\}\{num\}\{slash\}\{num\}\newline
CONSTINIT $\rightarrow$ \{CONSTLIST\}\newline
CONSTLIST $\rightarrow$ \{CONST\}\newline
CONSTLIST $\rightarrow$ \{CONSTLIST\}\{CONST\}\newline
CONST $\rightarrow$ \{const\}\{VALUE\}\{equals\}\{VALUE\}\{semicolon\}\newline
VOICELIST $\rightarrow$ \{VOICE\}\newline
VOICELIST $\rightarrow$ \{VOICELIST\}\{VOICE\}\newline
VOICE $\rightarrow$ \{voicebegin\}\{leftpar\}\{VALUE\} \{rightpar\}\{leftcurl\}\{VOICECONTENT\}\{rightcurl\}\newline
VOICECONTENT $\rightarrow$ \{COMPAS\}\newline
VOICECONTENT $\rightarrow$ \{COMPASLOOP\}\newline
VOICECONTENT $\rightarrow$ \{VOICECONTENT\}\{COMPAS\}\newline
VOICECONTENT $\rightarrow$ \{VOICECONTENT\}\{COMPASLOOP\}\newline
COMPASLOOP $\rightarrow$ \{loopbegin\}\{leftpar\}\{VALUE\}\ {rightpar\}\{leftcurl\}\{COMPASLIST\}\{rightcurl\}\newline
COMPASLIST $\rightarrow$ \{COMPAS\}\newline
COMPASLIST $\rightarrow$ \{COMPASLIST\}\{COMPAS\}\newline
COMPAS $\rightarrow$ \{compasbegin\}\{leftcurl\}\{NOTELIST\}\{rightcurl\}\newline
NOTELIST $\rightarrow$ \{NOTE\}\newline
NOTELIST $\rightarrow$ \{SILENCE\}\newline
NOTELIST $\rightarrow$ \{NOTELIST\}\{NOTE\}\newline
NOTELIST $\rightarrow$ \{NOTELIST\}\{SILENCE\}\newline
NOTE $\rightarrow$ \{notebegin\}\{leftpar\}\{notename\}\{alter\}\{comma\}\{VALUE\} \{comma\}\{shape\}\{punto\}\{rightpar\}\{semicolon\}\newline
NOTE $\rightarrow$ \{notebegin\}\{leftpar\}\{notename\}\{alter\}\{comma\}\{VALUE\} \{comma\}\{shape\}\{rightpar\}\{semicolon\}\newline
NOTE $\rightarrow$ \{notebegin\}\{leftpar\}\{notename\}\{comma\}\{VALUE\} \{comma\}\{shape\}\{punto\}\{rightpar\}\{semicolon\}\newline
NOTE $\rightarrow$ \{notebegin\}\{leftpar\}\{notename\}\{comma\}\{VALUE\} \{comma\}\{shape\}\{rightpar\}\{semicolon\}\newline
SILENCE $\rightarrow$ \{silencebegin\}\{leftpar\}\{shape\}\{rightpar\}\{semicolon\}\newline
SILENCE $\rightarrow$ \{silencebegin\}\{leftpar\}\{shape\}\{punto\}\{rightpar\}\{semicolon\}\newline
VALUE $\rightarrow$ \{cname\}\newline
VALUE $\rightarrow$ \{num\}\newline

\subsection{Tokens}
\begin{itemize}
\item TEMPOBEGIN: \# tempo

\item CONST: const

\item EQUALS: $=$

\item SEMICOLON: ;

\item VOICEBEGIN: voz

\item LEFTPAR: (

\item RIGHTPAR: )

\item LEFTCURL: \{

\item RIGHTCURL: \}

\item COMPASHEADERBEGIN: \# compas

\item COMPASBEGIN: compas

\item LOOPBEGIN: repetir

\item SLASH: $/$

\item NOTEBEGIN: nota

\item SILENCEBEGIN: silencio

\item PUNTO: .

\item ALTER: $+|-$

\item SHAPE: $blanca|negra|redonda|semicorchea|corchea|semifusa|fusa$

\item NOTENAME: $do|re|mi|fa|sol|la|si$

\item COMMA: ,

\item CNAME: $(([a-z]|[A-Z])([0-9]|[a-z]|[A-Z])*)$

\item NUM: $([0]|[1-9][0-9]*)$

\end{itemize}

\newpage

\section{Tests}
\lstset{
  frame=none,
  xleftmargin=2pt,
  stepnumber=1,
  numbers=left,
  numbersep=5pt,
  numberstyle=\ttfamily\tiny\color[gray]{0.3},
  belowcaptionskip=\bigskipamount,
  captionpos=b,
  escapeinside={*'}{'*},
  language=prolog,
  tabsize=2,
  emphstyle={\bf},
  commentstyle=\it,
  stringstyle=\mdseries\rmfamily,
  showspaces=false,
  keywordstyle=\bfseries\rmfamily,
  columns=flexible,
  basicstyle=\small\sffamily,
  showstringspaces=false,
  morecomment=[l]\%,
}

\subsection{Tests con fallas}

\subsubsection{Test 1: Tiene compaces con distinta duración}

\begin{lstlisting}
#tempo negra 30
#compas 3/4

const oct1 = 2;
const oct2 = 6;
const oct3 = 1;

// Instrumentos
const flauta = 51;

voz (flauta)
{
  compas
  {
    nota(do, oct3, blanca.);
    nota(re, oct1, redonda);
  }

  compas
  {
    nota(mi, oct2, blanca);
    nota(la, oct1, negra);
  }
}
\end{lstlisting}
\vspace{5mm}

\subsubsection{Test 2: Tiene voces con compaces de distinta duración}

\begin{lstlisting}
#tempo negra 120
#compas 2/2

const oct1 = 5;
const oct2 = 2;
const oct3 = 4;

// Instrumentos
const violin = 20;
const guitarra = 12;

voz (violin)
{
  compas
  {
    nota(do, oct3, blanca.);
    nota(re, oct1, negra);
  }

  compas
  {
    nota(mi, oct2, blanca.);
    nota(la, oct1, negra);
  }
}

voz (guitarra)
{
  compas
  {
    nota(do, oct3, fusa);
    nota(re, oct1, semifusa.);
  }

  compas
  {
    nota(mi, oct2, fusa);
    nota(la, oct1, semifusa.);
  }
}
\end{lstlisting}
\vspace{5mm}

\subsubsection{Test 3: Constante que apunta a una constante no definida}

\begin{lstlisting}
#tempo negra 60
#compas 1/1

const oct1 = 3;
const oct2 = ConstanteTrucha;

// Instrumentos
const bajo = 20;

voz (bajo)
{
  compas
  {
    nota(do, oct1, blanca.);
    nota(re, oct1, negra);
  }

  compas
  {
    nota(mi, oct2, blanca.);
    nota(la, oct2, negra);
  }
}
\end{lstlisting}
\vspace{5mm}

\subsubsection{Test 4: Constante definida circularmente}

\begin{lstlisting}
#tempo negra 60
#compas 2/8

const oct1 = 3;
const oct2 = 5;

// Instrumentos
const bajo = 20;
const malPensado = malPensado;

voz (bajo)
{
  compas
  {
    nota(do, oct1, corchea.);
    nota(re, oct1, semicorchea);
  }

  compas
  {
    nota(mi, oct2, semicorchea);
    nota(la, oct2, corchea.);
  }
}
\end{lstlisting}
\vspace{5mm}

\subsection{Tests correctos}

\subsubsection{Test 1: Simple}

\begin{lstlisting}
#tempo negra 30
#compas 2/4

const oct1 = 2;
const oct2 = 6;
const oct3 = 1;

// Instrumentos
const flauta = 51;

voz (flauta)
{
  compas
  {
    nota(do, oct3, negra);
    nota(re, oct1, negra);
  }

  compas
  {
    nota(mi, oct2, blanca);
  }
}
\end{lstlisting}
\vspace{5mm}

\subsubsection{Test 2: Con varias voces}

\begin{lstlisting}
#tempo negra 120
#compas 3/4

const oct1 = 2;
const oct2 = 6;
const oct3 = 1;
const oct4 = 3

// Instrumentos
const piano = 65;
const violin = 31;

voz (piano)
{
  compas
  {
    nota(sol, oct3, blanca);
    nota(fa+, oct2, negra);
  }

  compas
  {
    nota(mi, oct2, negra.);
    nota(fa, oct4, corchea);
    nota(mi, oct2, negra);
  }
  compas
  {
    silencio(negra);
    nota(sol-, oct1, negra);
    nota(sol-, oct1, negra);
  }
}

voz (violin)
{
  compas
  {
    nota(la, oct1, blanca.);
  }
  
  compas
  {
    nota(mi, oct2, negra.);
    nota(fa, oct4, corchea);
    nota(mi, oct2, negra);
  }
  compas
  {
    silencio(semicorchea);
    nota(mi, oct1, semicorchea);
    nota(sol, oct4, corchea);
    nota(sol, oct4, blanca);
  }
}
\end{lstlisting}
\vspace{5mm}

\subsubsection{Test 3: Con repeticiones}

\begin{lstlisting}
#tempo negra 120
#compas 2/4

const oct1 = 2;
const oct2 = 6;
const oct3 = 1;
const oct4 = 3

// Instrumentos
const flauta = 10;
const violin = 31;

voz (flauta)
{
  repetir (5)
  {
  	compas
  	{
    		nota(sol, oct3, negra);
    		nota(fa+, oct2, negra);
  	}

  	compas
  	{
    		nota(mi, oct2, negra.);
    		nota(fa, oct4, corchea);
  	}
  	compas
  	{
    		nota(sol, oct1, blanca);
  	}
  }
}

voz (violin)
{
  compas
  {
    nota(la, oct1, blanca);
  }
  
  compas
  {
    nota(mi, oct2, negra.);
    nota(fa, oct4, corchea);
    nota(mi, oct2, negra);
  }
  compas
  {
    silencio(semifusa);
    nota(si+, oct1, semifusa);
    nota(si, oct4, semifusa);
    nota(fa, oct4, semifusa);
    nota(sol, oct4, semicorchea);
    nota(mi-, oct4, corchea);
    nota(re, oct4, negra);
  }
}
\end{lstlisting}
\vspace{5mm}
\newpage

\section{Manual del programa}
\subsection{Modo de uso}
Línea de ejecución: ./musileng entrada.mus salida.txt

\subsubsection{Reglas para evitar posibles errores}
Se aceptan archivos de entrada que contengan:
\begin{itemize}
\item Todos los compases deben tener la misma duración al sumar la duración de sus notas y/o silencios.
\item Todas las voces deben tener la misma cantidad de compases.
\item No deben existir constantes indefinidas o definidas circularmente. Ejemplo constante no definida: $const$ $eval = hola;$ (hola jamás se definió). Ejemplo constante definida circularmente: $const$ $eval1 = eval2$ $;$ $const$ $eval2 = eval1$ $;$.
\item La suma de la duración de cada compas debe ser igual a $num1/num2$, donde $num1$ y $num2$ son los números definidos en $\#compas$ $num1/num2$ en el encabezado.
\item Una constante definida como instrumento sólo acepta valores del $1$ al $127$.
\item El valor colocado en 'repetir' debe ser mayor a $0$.
\item Una constante definida como octava sólo acepta valores del $1$ al $9$.
\item No debe haber una constante definida más de una vez.
\item En $\#compas$ $num1/num2$ del encabezado, $num2$ debe ser un número que corresponda al de las figuras definidas, para corroborar puede ver la página 2 de \url{www.dc.uba.ar/materias/tl/2015/c1/tp2-enunciado-compositor-musical/at_download/file} en la tabla Figura y Valor numérico asociado.
\end{itemize}
En caso contrario que no se respete lo mencionado anteriormente, nuestro programa especificará el error cometido para que pueda solucionarlo.

Para más información sobre los archivos de entrada y salida puede mirarse el siguiente pdf \url{www.dc.uba.ar/materias/tl/2015/c1/tp2-enunciado-compositor-musical/at_download/file}.

\subsection{Requerimientos necesarios para ejecutar}
\begin{itemize}
\item Programa: Python
\item Versión: 2.7
\end{itemize}

%\begin{itemize}
%\item En la primera línea: #tempo FIGURA N (Falta desarrollar)
%\item En la segunda línea: #compas N/M
%\item En las siguientes líneas definiremos las constantes de las octavas, por ejemplo: "$const oct1 = 6;$"
%\item Luego de las octavas definiremos las constantes de instrumentos, por ejemplo: "$const piano = 31;$"
%\item Más adelante seguirán una lista de voces, a las cuales se les asignarán su respectivo instrumento definido anteriormente
%\item En cada voz definiremos una lista de compases o un loop de compases.
%\item Un loop de compases se define mediante un repetir, el cual cuenta con un número entero que indicará la cantidad de veces que se repetirá, seguido de una lista de compases que son los que se repetirán la cantidad de mencionada en el número entero.
%\item Un compás tiene una lista de notas o/y silencios.
%\item 
%\end{itemize}
\newpage

\section{Conclusiones}
\begin{itemize}
\item PLY es una herramienta muy útil, facilita el parseo y nos permite, de manera mucho más corta y sencilla, realizar las diferentes tareas para nuestro lenguaje.
\item Lo más complicado fue la parte del lexer, específicamente en la parte que definimos los tokens. Porque como explicamos en la descripción necesitabamos establecer un orden y para eso tuvimos que, entre otras cosas, entender como funcionaba y probar la clase re (regular expression) de python.
\item Los temas aportados en las clases, como gramática de atributos, TDS, parser, gramática LALR fueron útiles para poder resolver los problemas presentados.

\end{itemize}


\end{document}

